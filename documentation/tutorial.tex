\documentclass[letterpaper,11pt]{article}

\usepackage{fullpage}
\usepackage{amsmath}
\usepackage{amsfonts}
\usepackage{url}

\usepackage{listings}
\lstset{
  language=Lisp,
  extendedchars=true,
  basicstyle=\ttfamily,
  stringstyle=\ttfamily\slshape,
  commentstyle=\slshape,
%  numbers=left,
%  stepnumber=5,
%  numbersep=6pt,
%  numberstyle=\footnotesize,
  breaklines=true,
%  frame=single,
  columns=fullflexible,
  literate={-}{}{0\discretionary{-}{}{-}},
}

\begin{document}

\title{The \lstinline!affi! package}
\author{Tam\'as K Papp\thanks{E-mail: tpapp@princeton.edu}}
\date{\today}
\maketitle

\begin{abstract}
  This is a tutorial for the \lstinline!affi! (affine indexing)
  package, that provides convenience functions for traversing slices
  of arrays, optionally with index permutations and other convenient
  transformations.  A driver clause for \lstinline!iterate! is
  provided, along with \lstinline!map-subarray!, a simple yet powerful
  function for mapping arrays into each other using affine indexes.
\end{abstract}

\section{Introduction}
\label{sec:introduction}

Even though this is not widely known, Common Lisp is an amazing
language for numerical computation.  The language constructs
themselves are very powerful, and efficient compilers exist that allow
the user to run his code very quickly even without tedious
optimization, while allowing for the possibility of the latter when
the need arises.  Also, its interactive nature makes the language
ideal for debugging glitches in the algorithm or exploring problems
quickly.

However, Common Lisp lacks a few features that users of languages like
Matlab, R or even Fortran are used to.  One of these is the possibily
of ``slicing'' arrays: for example, treating a column of a matrix as a
vector, a rectangular submatrix of a matrix as another matrix when
passing to another function, assigning rectangular sections of arrays
to each other and performing operations on them with ease, with
appropriate syntactic sugar and usually corresponding fast code.

This package is an attempt to fill this gap and implement a similar
--- but not identical --- feature in Common Lisp called affine
indexes.  Affine indexes and appropriate helper functions allow CL
users to copy, modify, transpose rectangular slices of arrays with
ease.  Instead of implementing syntactic sugar like \verb!3:5! for
elements from 3 to 5, I decided that a functional approach would be
better.  Thus instead of doing the tasks with specialized functions,
affine indexes allow the user to generate a \emph{walker}, which is
essentially a function that yields successive indexes for the flat
vector that holds the elements of an array.
Section~\ref{sec:affine-indexes} presents affine indexes and walkers, and
Section~\ref{sec:operations} explains some predefined operations on
affine indexes (including subranges).  Section~\ref{sec:lstinl-subarr-an}
discusses useful functions constructed with affine indexes.

\section{Affine indexes and walkers}
\label{sec:affine-indexes}

An array of rank $n$ can be conceptualized as a function that maps a
set of integer subscripts $(s_1,\dots,s_n)$, which are constrained by
the dimensions of the array, to an element.  It is useful to imagine
that the elements in the array are stored in a flat vector with a
single index $i$, and think of this function as a composite of two
other functions: one $(s_1,\dots,s_n)\mapsto i$ which maps the
subscripts to the index in the flat vector, and another which
retrieves the element of the given index $i$ from this vector.  In
Common Lisp, \lstinline!aref! does these two things in a single step,
but like above, its operation can be broken up into two stages,
implemented as the functions \lstinline!array-row-major-index! and
\lstinline!row-major-aref!.

Affine indexes are nothing more than a class of integer functions that
are more general than those used for row-major arrays.  Consider a
positive integer $n$ (which we will call the \emph{rank}), and a
\emph{domain}
\begin{equation*}
  D(d_0,\dots,d_{n-1}) = \bigl([0,1,\dots,d_0)\cap \mathbb{N}\bigr)
  \times \dots \times \bigl([0,1,\dots,d_{n-1})\cap \mathbb{N}\bigr)
\end{equation*}
where $d_i$ is the \emph{size} of dimension $i$.  An affine index is a
mapping $a: D \to \mathbb{Z}$, defined as
\begin{equation*}
  a(s_0,\dots, s_{n-1}) = C + c_0s_0 + \dots + c_{n-1}s_{n-1}
\end{equation*}
where $C\in\mathbb{Z}$ is called the \emph{constant}, and
$c_i\in\mathbb{Z}, i=0,\dots,n-1$ are the \emph{coefficients}.

An affine index is characterized by its domain, constant and
coefficient.  It is easy to see that affine indexes are quite general and
can be used to implement row-major (or column-major, for that matter)
indexing: for example, the $0+12s_0+4s_1+1s_2$ would index an array
with dimensions \lstinline!#(2 3 4)! (with the appropriate domain).
Moreover, affine indexes allow us to index rectangular sub-arrays, or
even reverse elements or permute indexes (transposing a matrix is a
special case of the latter).

Before I discuss functions in this package, two general remarks are in
order.  First, all functions in the package are properly documented,
so you can read their documentation string for a complete description
--- examples in this tutorial are just meant to whet your appetite and
make you aware of features, and are not meant to be extensive
descriptions.  Second, the function names are deliberately kept short,
because I intend that they are used with their namespace prefix and
not imported to the current namespace using \lstinline!use-package!
--- hence in your code, you would write \lstinline!affi:drop! instead
of \lstinline!drop!.  Examples below are run from the \lstinline!affi!
namespace for simplicity, but generally you should use another one
(\lstinline!cl-user:!, your own package, etc).

You can create an affine index conforming to a list if dimensions
characterizing a row-major array or the array itself using
\lstinline!make-affi!:
\begin{lstlisting}
AFFI> (make-affi '(2 3 4))
#<AFFI domain #(2 3 4), const 0, coeff #(12 4 1) {B0C4D61}>
AFFI> (make-affi (make-array '(5 6 7)))
#<AFFI domain #(5 6 7), const 0, coeff #(42 7 1) {B3B8501}>
\end{lstlisting}

Or should you find yourself in need of interfacing with some vile
language that uses column-major indexing, call
\lstinline!make-affi-cm! on a list of indexes:
\begin{lstlisting}
AFFI> (make-affi-cm '(2 3 4))
#<AFFI domain #(2 3 4), const 0, coeff #(1 2 6) {B7340F1}>
\end{lstlisting}

How can you use affine indexes?  If you just want to calculate the
index of a single element, you can use
\begin{lstlisting}
AFFI> (defparameter *a* (make-affi '(2 3 4)))
AFFI> (calculate-index *a* #(0 1 2))
6
\end{lstlisting}

But when you are traversing an array, it is better to use a
\emph{walker}.  A walker is simply a function that traverses the
subscripts in a particular (lexicographic, with rightmost subscripts
changing the fastest) order, and returns the index corresponding to
each:
\begin{lstlisting}
AFFI> (defparameter *w1* (make-walker (make-affi '(2 2))))
AFFI> (funcall *w1*)
0
AFFI> (funcall *w1*)
1
AFFI> (funcall *w1*)
2
AFFI> (funcall *w1*)
3
AFFI> (funcall *w1*)
NIL
\end{lstlisting}
On each call, it returns the \emph{current} index, and \emph{then}
increments its internal counter.  Once it runs out of elements, it
will just return \lstinline!nil!.  \lstinline!make-walker! actually
returns two functions, and the second one allows you to query the
index without incrementing it.

Of course, a row-major walker is not a particularly interesting one,
as it justs lists consecutive integers.\footnote{This is of course
  recognized, and walkers are optimized to recognize contiguous blocks
  of integers. [FIXME: not yet, still working on it]} Using the
convenience function \lstinline!test-walker!,\footnote{This functions
  displays the indexes returned by a walker until it encounters
  \lstinline!nil!.  You can also call it directly on an affine index.}
this is how a column-major walker would traverse the indexes:
\begin{lstlisting}
AFFI> (test-walker (make-walker (make-affi-cm '(2 3))))
0 2 4 1 3 5 
\end{lstlisting}

Finally, there are a few convenience functions: \lstinline!size!
returns the number of integers indexed, \lstinline!range! returns the
smallest and largest integers indexed, and \lstinline!rank! returns
the rank.

When using two walkers simultaneously, it is a good idea to check if
they are \emph{conformable}.  You can do that using
\lstinline!check-conformability!, which recognizes three types of
conformability (in order of decreasing strictness):
\begin{itemize}
\item  \lstinline!strict! requires that the two domains are
exactly the same
\item \lstinline!dropped! checks if they are the same when we drop
dimensions of size 1
\item \lstinline!size! just checks the size of the two ranges.
\end{itemize}
\lstinline!dropped! is usually a good compromise between catching
errors and convenience, it still allows you to treat a $6\times1$
matrix and a 6-element vector as conformable (\lstinline!strict! would
not), but would catch obvious errors like trying to walk the former
matrix alogn with a 7-element vector.

For convenience, there is an \lstinline!iterate! driver for affine
indexes:\footnote{I prefer to put a \lstinline!:! before keywords
  other than the first one when using iterate, you can just write\\
  \lstinline!(for i in-affi (make-affi '(2 3)))! if you prefer.}
\begin{lstlisting}
(iter
  (for i :in-affi (make-affi '(2 3)))
  (collecting i)  ;; evaluates to (0 1 2 3 4 5)
\end{lstlisting}


\lstinline!affi! instances are \emph{not} meant to be modified
directly: all utility functions below leave the original argument
intact and create a new instance.  You can safely assume that an
\lstinline!affi! instance is immutable, and the slots are not exported
from the package.  Nevertheless, they can be accessed
\lstinline!get-const!, \lstinline!get-coeff! and
\lstinline!get-domain!, the latter two makes copies of the internal
vectors.

\section{Operations}
\label{sec:operations}

Of course, affine indexes become useful when we consider
transformations on them.  The \lstinline!affi! package has the
following predefined transformations:

\lstinline!permute!, which allows us to shuffle indexes around, for
example, \lstinline!(permute affi '(1 0))! is equivalent to
transposing a matrix (actually, \lstinline!transpose! is defined too),
but works for affine indexes of any rank:
\begin{lstlisting}
AFFI> (defparameter *affi* (make-affi '(2 3 4)))
AFFI> *affi*
#<AFFI domain #(2 3 4), const 0, coeff #(12 4 1) {BCEBFF9}>
AFFI> (permute *affi* '(1 2 0))
#<AFFI domain #(3 4 2), const 0, coeff #(4 1 12) {A903111}>
\end{lstlisting}

\lstinline!drop!, which allows the elimination of degenerate
dimensions (of size $1$).  It has an optional argument which is
supposed to contain the list of dimensions that are considered for
elimination:
\begin{lstlisting}
AFFI> (drop (make-affi '(1 1 2 3)))
#<AFFI domain #(2 3), const 0, coeff #(3 1) {A9D5441}>
AFFI> (drop (make-affi '(1 1 2 3)) '(1))
#<AFFI domain #(1 2 3), const 0, coeff #(6 3 1) {AA04371}>
\end{lstlisting}
When this last argument is \lstinline!t! (the default), all degenerate
dimensions will be eliminated.

Finally, the most powerful operation is \lstinline!subrange!, which
constrains and/or transforms an affine index given a list of range
specifications, one for each dimensions.

One way to specify a range is using a two-element list: for example,
\lstinline!(3 5)! denotes the subscripts from 3 to 5, inclusive.
Negative numbers are counted from the dimension: if a subscript is
constrained to be below 6, -3 will denote 3.  You can mix positive and
negative subscripts, as in \lstinline!(3 -1)!.  If the two subscripts
are in decreasing order (after being converted to nonnegative
integers), the indexes will be \emph{reversed} along this dimension.

Also, there are some shortcuts: specifying a single integer
\lstinline!n! is equivalent to \lstinline!(n n)!, selecting a single
subscript.  \lstinline!all! select all subscripts, and \lstinline!rev!
selects them in reverse order.  Some examples:
\begin{lstlisting}
AFFI> (defparameter *affi* (make-affi '(4 5)))
AFFI> (test-walker *affi*)
0 1 2 3 4 5 6 7 8 9 10 11 12 13 14 15 16 17 18 19 
AFFI> (test-walker (subrange *affi* '(all rev)))
4 3 2 1 0 9 8 7 6 5 14 13 12 11 10 19 18 17 16 15 
AFFI> (test-walker (subrange *affi* '((1 -1) 2)))
7 12 17 
AFFI> (test-walker (subrange *affi* '((-1 1) 2)))
17 12 7 
\end{lstlisting}

\section{\lstinline!map-subarray!}
\label{sec:lstinl-subarr-an}

The function
\begin{lstlisting}
(defun map-subarray (source target 
		      &key source-range target-range permutation
                      (drop-which t)
		      (conformability 'dropped) (key #'identity)
		      (target-element-type (array-element-type source)))
\end{lstlisting}
combines the above transformations and allows us to copy ranges from
one array to another, optionally permuting the indexes.

If \lstinline!target! is \lstinline!nil!, a new array is created with
element-type \lstinline!target-element-type! and is returned by the
function (in this case, you can't specify \lstinline!target-range!,
and degenerate dimensions will be dropped according to
\lstinline!drop-which!). Otherwise, conformability is checked between
the two affine indexes (which are calculated internally using the
given ranges and permutation) using
\lstinline!conformability!.\footnote{See
  \lstinline!check-conformability! three kinds of conformability
  recognized by \lstinline!affi!.} \lstinline!key! is
called on each element before copying and the result is copied in the
target array.

Some examples:
\begin{lstlisting}
AFFI> *m* ;; the original array
#2A((0 1 2 3) (4 5 6 7) (8 9 10 11))
AFFI> (map-subarray *m* nil :permutation '(1 0)) ;; transpose
#2A((0 4 8) (1 5 9) (2 6 10) (3 7 11))
AFFI> (map-subarray *m* nil :source-range '(all 1)) ;; 2nd column
#(1 5 9)
AFFI> (map-subarray *m* nil :source-range '(all 1) :drop-which nil)
#2A((1) (5) (9))
AFFI> (map-subarray *m* nil :source-range '((1 2) rev))
#2A((7 6 5 4) (11 10 9 8))
\end{lstlisting}

If you check the source code of \lstinline!map-subarray!, you will
notice that it is very straightforward.  While the function is useful
as it is, the tools in this package make writing similar functions
easy.

\end{document}
